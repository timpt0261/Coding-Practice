\documentclass[11pt]{article} % Document type
\usepackage{listings}
\usepackage{color}
\usepackage{blindtext}
\usepackage{hyperref}
\definecolor{dkgreen}{rgb}{0,0.6,0}
\definecolor{gray}{rgb}{0.5,0.5,0.5}
\definecolor{mauve}{rgb}{0.58,0,0.82}

\lstset{frame=tb,
  language=Python,
  aboveskip=3mm,
  belowskip=3mm,
  showstringspaces=false,
  columns=flexible,
  basicstyle={\small\ttfamily},
  numbers=none,
  numberstyle=\tiny\color{gray},
  keywordstyle=\color{blue},
  commentstyle=\color{dkgreen},
  stringstyle=\color{mauve},
  breaklines=true,
  breakatwhitespace=true,
  tabsize=3
}
\title{Assignment 5 \\
    \large Public Key Cryptography \\
    \textbf{DESIGN.pdf}}
\author{Reuben T. Chavez}
\date{\today} % Sets the date to \today, or any date you specify
\begin{document}
\maketitle % Start the document

\pagebreak
\section*{Pseudeocode}
\begin{flushleft}
\begin{lstlisting}

#Libaries
import randstate
import numtheory
import rsa
import stdlib
import bool
import stdint

# Psuedocode for keygen.c

## Usage ##
function  usage  is
    input: executable
    output: void

    print(
    "Synopsis of Keygen\n"

    "Usage of KeyGen\n"

    "Options for KeyGen\n"
    )

function main is
    input : argument count argc and argument vector argv
    output: zero to exit program
    
    opt <- 0
    Set bits as a unsigned 64 bit number
    verbose <- false
    iterations <- 50
    Set public rsa file to be read
    Set private rsa file to be read
    Set seed to explicit starting point
    Set random to the created seed
    while getting commands from command line do
        switch command:
            case bits:
                set bits to the user's argument
                break
            case iteration :
                set iteration to the command line arguments
                break
            case public file :
                if given files exists
                    pbfile <- users's argument
                    break
                Print that the given file does not exist and end program
            case private file :
                if given files exists
                    pvfile <- users's argument
                    break
                Print that the given file does not exist and end program
            case seed
                seed <- User's argument
                Inintialize reandom to staart at given seed
                break
            case verbose:
                verbose <- true
                break
            default help:
                prints usage and ends program

    Check the both the public file and private file have file key permisiion to 600

    Ininitialize random state with given seed
    
    Create public key with function in rsa library

    Create private key with function in rsa library

    Get current user's name in the /home/username path
    
    Convert username to integer of base 62 and use rsa sign in library

    if verbose is true:
        print {Username
               Signature
               First Large Prime 
               Second Large Prime
               Public Modulus
               Public Exponent
               Private Key 
                } 
    Write public modulus , public exponent, siganuture , and username into public file

    Write public modulus, private key into private file

    Clear all given files and mpz intergers
    return 0
\end{lstlisting}

\begin{lstlisting}

#Libaries
import numtheory
import rsa
omport randstate
import stdlib
import bool
import stdint

# Psuedocode for encrypt.c

## Usage ##
function  usage  is
    input: executable
    output: void

    print(
    "Synopsis of encrypt\n"

    "Usage of encrypt\n"

    "Options for encrypt\n"
    )

function main is
    input : argument count argc and argument vector argv
    output: zero to exit program
    
    intialize opt to 0
    initialize input to standard input
    initialize output object to standard output
    initialize pvfile object to private file 
    initalize verbose as false

    while getting commands from command line do
        switch command:
            case i:
                if file exists:
                    input is set to read file
                    break
                print that the file does not exist
                stop running
            case o :
                if file exists:
                    output is set to read file 
                    break
                print that the file doe not exist
                stop running
            case n :
                if file exists:
                    pvfile is set to read file 
                    break
                print that the file doe not exist
                stop running
            case verbose:
                set verbose to true
            default help:
                prints usage and ends program
    
    Initalize mpz-t varibles that store the public modulus and public exponent

    Read given file the set the private key and public modulus

    if verbose:
        print(
             The public modulus
             The public exponent   
             )
    decrypt the give input file to the given output file with the public modulus and public exponent
    
    clear the mpz-t varibles
    close all opened files
    return 0
\end{lstlisting}

\begin{lstlisting}

#Libaries
import numtheory
import rsa
import randstate
import stdlib
import bool
import stdint

# Psuedocode for decrypt.c

## Usage ##
function  usage  is
    input: executable
    output: void

    print(
    "Synopsis of decrypt\n"

    "Usage of decrypt\n"

    "Options for decrypt\n"
    )

function main is
    input : argument count argc and argument vector argv
    output: zero to exit program
    
    intialize opt to 0
    initialize input to standard input
    initialize output object to standard output
    initialize pvfile object to private file 
    initalize verbose as false

    while getting commands from command line do
        switch command:
            case i:
                if file exists:
                    input is set to read file
                    break
                print that the file does not exist
                stop running
            case o :
                if file exists:
                    output is set to read file 
                    break
                print that the file doe not exist
                stop running
            case n :
                if file exists:
                    pvfile is set to read file 
                    break
                print that the file doe not exist
                stop running
            case verbose:
                set verbose to true
            default help:
                prints usage and ends program
    
    Initalize mpz-t varibles that store the public modulus and public exponent
    Initilez username to NULL

    Read given file the set the public modulus, public exponent, signature, and private key

    if verbose:
        print(
             The username
             The signature
             The public modulus
             The public exponent
             The private key   
             )
    encrypt the give input file to the given output file with the public modulus and public exponent
    
    clear the mpz-t varibles
    close all opened files
    return 0
\end{lstlisting}


\begin{lstlisting}
#Pseudocode for rsa.c


#Libaries
import numtheory
import rsa
import randstate

import stdbool
import stdint
import stdio
import stdlib
import gmp
import gmp
import math


function lcm is :
    input: mpzt output, mpzt a mpzt b
    output nothing

    Initialize varibles for numerator and denominator
    
    Set numertor to equal to abslut value of the a times b
    
    Set denominator to the greatest common divisior of a and b

    Set output to numerator divided by denominator
    
    clear mpzt varibles

function rsa-make-pub
    input: mpzt p, mpzt, q , mpzt n. mpzt e, nbits , iterations
    output: nothing
    
    make p equal to prime number that is in range of nbit/4 to 3*nbits/r
    make q equal to rest of the nbits

    

\end{lstlisting}

\begin{lstlisting}
#Pseudocode for randstate.c

\end{lstlisting}


\begin{lstlisting}
#Pseudocode for numbtheory.c

\end{lstlisting}

\end{flushleft}


\end{document}
\grid

