\documentclass[11pt]{article} % Document type
\usepackage{listings}
\usepackage{color}
\usepackage{blindtext}
\usepackage{hyperref}
\definecolor{dkgreen}{rgb}{0,0.6,0}
\definecolor{gray}{rgb}{0.5,0.5,0.5}
\definecolor{mauve}{rgb}{0.58,0,0.82}

\lstset{frame=tb,
  language=Python,
  aboveskip=3mm,
  belowskip=3mm,
  showstringspaces=false,
  columns=flexible,
  basicstyle={\small\ttfamily},
  numbers=none,
  numberstyle=\tiny\color{gray},
  keywordstyle=\color{blue},
  commentstyle=\color{dkgreen},
  stringstyle=\color{mauve},
  breaklines=true,
  breakatwhitespace=true,
  tabsize=3
}
\title{Assignmient 7 \\
    \large Author Identification \\
    \textbf{DESIGN.pdf}}
\author{Reuben T. Chavez}
\date{\today} % Sets the date to \today, or any date you specify
\begin{document}
\maketitle % Start the document

\pagebreak
\section*{Pseudeocode}
\begin{flushleft}

\begin{lstlisting}

# Psuedocode for identify.c

function  usage  is
    input: executable
    output: void


function main is
    input : argument count argc and argument vector argv
    output: zero to exit program

    - Perform getop operators to determine:

    - 
        
\end{lstlisting}

\begin{lstlisting}

\end{lstlisting}


\begin{lstlisting}
#Pseudocode for Nodes

##Libraries

import stdint.h

define type Node;

Initialize struct Node with:
    - char pointer to word
    - unsigened inter of 32 bit to count

Function Node Create:
    input: char pointer
    output: a pointer to a Node type
    
    Allocate space for Node n with a size of Node 
    Allocate space for word in Node
    Set Node's word to copy of input
    return Node n



Function Node Delete:
    Input: Double pointer n
    Output: None since functioin is void
    
    - free word since space was allocated for it
    - free contents in n
    - set to null

Function print node:
    input: pointer to node
    output: Nothing function is void

    print the data item withn the pointer node
\end{lstlisting}


\begin{lstlisting}
#Pseudocode for pq.c

#Libraries 
import node
import stdbool
import stdint


struct PriorityQueue
    - contains head
    - contains tail
    - contains capacity
    - contains Node array

Function Create Priority Queue:
    Input: An unsigned intger of 32 bits
    Output: Priorit Queue pointer
    
    - Allocate space for a Priority Queue pointer
    - Initialize head, tail, capacity, and the Node array if the pq is not NULL

Function Insertion Sort:
    Input: A Priority Queue and Node
    Output: Nothing function is void
    - For iteration of Priorty queue:
        - set j to current index
        - create temp of current index in PQ array
        - While j is greater than 0 and the temp is greater than the last index
            - set array at index j to the last insex
            - subtract j by 1
        - set the Priorty Queue on index j to temp

Function pq delete:
    Input : Double pointer to Priority Queue
    Output : Nothing function is void

    - If the input is not null, free double pointer and set previous node to NULL

Function pq full:
    Input : Double pointer to Priority Queue
    Output: boolean
    
    - rturn that given pq is either full or not


Function pq empty:
    Input : Double pointer to Priority Queue
    Output: boolean
    
    - rturn that given pq is either empty or not


Function pq size:
    Input : Priorty Queue pointer
    Output: Unsigned 32 interger
    
    - return the the top node in pq 


Function enqueue:
    Input: Priorty Queue pointer and Node pointer
    Output: boolean

    - if Priorty Queue not null, 
        - if empty return false
        - add node to head of pq
        - Resort the tail node to the in the pq using a sorting algorthim
    -return true to signify that the pq was succefully enqued 

Function dequeue:
    Input: Priorty Queue double pointer and Node pointer
    Output: boolean

    - if Priorty Queue not null, 
        - if full return false
        - remove node from tail of pq
        - Resort the tail node to the in the pq using a sorting algorthim
        - sub top by 1
    -return true to signify that the pq was succefully enqued 

Function pq print:
    Input: Priorty Queue Node
    Output: Nothing the function is void
    
    - Print all items with pq  
\end{lstlisting}

\begin{lstlisting}
#Pseudocode for stack.c

#Libraries
import node
import stdbool
import stdint
import stdlib

Stack struct:
    - contains top
    - contains capacity
    - contains double pointer node array

Function stack create 
    Input: unsigned 32 bit integer 
    Output: Pointer to Stack

    - Allocate memory fro STACK object
    - Initilize items within Stack struct
    - return stack pointer

Function stack delete:
    Input: Stack pointer 
    Output: Nothing function is void

    - Delete specifed stack, and set previous stack to NULL

Function stack empty:
    Input : Stack pointer
    Output: boolean
    
    - return if the stack is empty
     
Function stack full:
    Input : Stack pointer
    Output: boolean
    
    - return if the stack is full

Function stack push:
    Input: Stack pointer and pointer to node
    Output: boolean
    
    - Check if the stack is not full, if its add more space to stack
    - Add stack top and set node pointer equal to 


Function stack pop:
    Input: Stack pointer and double pointer to node
    Output: boolean
\end{lstlisting}

\begin{lstlisting}
#Pseudocode for code.c

\end{lstlisting}
\begin{lstlisting}
#Pseudocode for huffman.c

\end{lstlisting}

\begin{lstlisting}
#Pseudocode for 

\end{lstlisting}
\end{flushleft}


\end{document}
\grid

